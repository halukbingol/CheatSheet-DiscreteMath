\documentclass[a4paper,10pt,landscape]{article}




%% ~~~~~~~~~~~~~~~~~~~~~~~~~~~~~~~~~~~~~~~ V page layout
%\usepackage{fancyhdr}
%\pagestyle{fancy}
%%\fancyhead{}
%%\fancyhead[RO,LE]{Thesis Title}
%%\fancyfoot{}
%%\fancyfoot[LE,RO]{\thepage}
%%\fancyfoot[LO,CE]{Chapter \thechapter}
%%\fancyfoot[CO,RE]{Author Name}
%
%\lhead{}
%\chead{}
%\rhead{\textbf{The performance of new graduates}}
%\lfoot{From: K. Grant}
%\cfoot{To: Dean A. Smith}
%\rfoot{\thepage}
%\renewcommand{\headrulewidth}{0.4pt}
%\renewcommand{\footrulewidth}{0.4pt}
%% ~~~~~~~~~~~~~~~~~~~~~~~~~~~~~~~~~~~~~~~ A




% ~~~~~~~~~~~~~~~~~~~~~~~~~~~~~~~~~~~~~~~ V page setup
	\pagestyle{empty}
	
%	\setlength{\oddsidemargin}{-1cm}
%	\setlength{\evensidemargin}{-1cm}
%	\setlength{\textwidth}{27cm}
%	\setlength{\textheight}{18.5cm}
%	
%	\setlength{\topmargin}{-1cm}
%	\setlength{\headheight}{0cm}
%	\setlength{\headsep}{0cm}

	\setlength{\oddsidemargin}{-1.5cm}
	\setlength{\evensidemargin}{-1.5cm}
	\setlength{\textwidth}{27.7cm}
	\setlength{\textheight}{19cm}
	
	\setlength{\topmargin}{-1.5cm}
	\setlength{\headheight}{0cm}
	\setlength{\headsep}{0cm}
	
	
	
	
	% Redefine section commands to use less space
	\makeatletter
	\renewcommand{\section}{\@startsection{section}{1}{0mm}%
	                                {-1ex plus -.5ex minus -.2ex}%
	                                {0.5ex plus .2ex}%x
	                                {\normalfont\large\bfseries}}
	\renewcommand{\subsection}{\@startsection{subsection}{2}{0mm}%
	                                {-1explus -.5ex minus -.2ex}%
	                                {0.5ex plus .2ex}%
	                                {\normalfont\normalsize\bfseries}}
	\renewcommand\subsubsection{\@startsection{subsubsection}{3}{0mm}%
	                                {-1ex plus -.5ex minus -.2ex}%
	                                {1ex plus .2ex}%
	                                {\normalfont\small\bfseries}}
	\makeatother
	
	% Define BibTeX command
	\def\BibTeX{{\rm B\kern-.05em{\sc i\kern-.025em b}\kern-.08em
	    T\kern-.1667em\lower.7ex\hbox{E}\kern-.125emX}}
	
	% Don't print section numbers
	%\setcounter{secnumdepth}{1}
	%\setcounter{secnumdepth}{0}
	
	\setlength{\parindent}{0pt}
	\setlength{\parskip}{0pt plus 0.5ex}
% ~~~~~~~~~~~~~~~~~~~~~~~~~~~~~~~~~~~~~~~ A




% ~~~~~~~~~~~~~~~~~~~~~~~~~~~~~~~~~~~~~~~ V title part
	\usepackage[iso]{datetime}
	\newcommand{\hbTimeStamp}{{\color{red}v\today T\currenttime}} % 

	\usepackage[colorlinks=true,linkcolor=red,urlcolor=blue,citecolor=red]%
		{hyperref}
% ~~~~~~~~~~~~~~~~~~~~~~~~~~~~~~~~~~~~~~~ A




% ~~~~~~~~~~~~~~~~~~~~~~~~~~~~~~~~~~~~~~~ V color
	\usepackage{color}
		\definecolor{darkred}{rgb}{0.8,0.1,0.1}
%		\definecolor{darkgreen}{rgb}{0,0.5,0}
%		\definecolor{darkblue}{rgb}{0,0,0.5}
		\definecolor{colorSep}{rgb}{0.8,0.8,0.8}
	\newcommand{\hSep}{{\color{colorSep}\hrule}}
% ~~~~~~~~~~~~~~~~~~~~~~~~~~~~~~~~~~~~~~~ A




% ~~~~~~~~~~~~~~~~~~~~~~~~~~~~~~~~~~~~~~~ V  AMS packages 
	\usepackage{amssymb}
	\usepackage{amsmath}
	\usepackage{amsfonts}
	\usepackage{amsthm}
	\newtheorem{thm}{Theorem}[section]
	%
	\newtheorem{cor}[thm]{Corollary}
	\newtheorem{lem}[thm]{Lemma}
	\newtheorem{prop}[thm]{Proposition}
	\newtheorem{ax}{Axiom}
	%
	\theoremstyle{definition}
	\newtheorem{defn}{Definition}[section]
	%
	\theoremstyle{remark}
	\newtheorem{rem}{Remark}[section]
	\newtheorem*{notation}{Notation}
	\newtheorem{exmp}{Example}[section] % @HB
% ~~~~~~~~~~~~~~~~~~~~~~~~~~~~~~~~~~~~~~~ A




% ~~~~~~~~~~~~~~~~~~~~~~~~~~~~~~~~~~~~~~~ V references
	\newcommand{\reffig}[1]{Fig.~\ref{#1}}
	\newcommand{\reftbl}[1]{Table~\ref{#1}}
	\newcommand{\refsec}[1]{Sec.~\ref{#1}}
	\newcommand{\refthm}[1]{Theorem~\ref{#1}}
	\newcommand{\reflem}[1]{Lemma~\ref{#1}}
	\newcommand{\refdef}[1]{Definition~\ref{#1}}
	\newcommand{\refeq}[1]{Eq.~\ref{#1}}
% ~~~~~~~~~~~~~~~~~~~~~~~~~~~~~~~~~~~~~~~ A




% ~~~~~~~~~~~~~~~~~~~~~~~~~~~~~~~~~~~~~~~ V
	\usepackage{cancel}
	\usepackage{multicol}
	\usepackage{enumerate}
% ~~~~~~~~~~~~~~~~~~~~~~~~~~~~~~~~~~~~~~~ A




% ~~~~~~~~~~~~~~~~~~~~~~~~~~~~~~~~~~~~~~~ V \xymatrix
	\usepackage[all]{xy}
		\CompileMatrices % \xymatrix
% ~~~~~~~~~~~~~~~~~~~~~~~~~~~~~~~~~~~~~~~ A




%% ~~~~~~~~~~~~~~~~~~~~~~~~~~~~~~~~~~~~~~~ V
%\usepackage{pstricks, pst-node}
%% ~~~~~~~~~~~~~~~~~~~~~~~~~~~~~~~~~~~~~~~ A




% ~~~~~~~~~~~~~~~~~~~~~~~~~~~~~~~~~~~~~~~ V h-lib
\newcommand{\hDed}[1]{\textcolor{darkred}{\textit{#1}}} % definition
\newcommand{\hDiff}{\ensuremath{\overset{\Delta}{\iff}}} % <-->
\newcommand{\hDeq}{\ensuremath{\ \triangleq} \ }  % =
\newcommand{\hDev}{\ensuremath{\ \overset{\Delta}{\equiv}}\ } % =
%
\newcommand{\hAbs}[1]{\left\lvert\, #1\, \right\rvert}  % |x|
\newcommand{\hCeil}[1]{\left\lceil\, #1\, \right\rceil} % ceil(x)
\newcommand{\hFloor}[1]	{\left\lfloor\, #1\, \right\rfloor}% floor(x)

% ~~~~~~~~~~~~~~~~~~~~~~~~~~~~~~~~~~~~~~~ A




% ~~~~~~~~~~~~~~~~~~~~~~~~~~~~~~~~~~~~~~~ V Set of Numbers
\newcommand{\hSoN}  {\ensuremath{\mathbb{N}}}      % set of Natural Numbers
\newcommand{\hSoNp} {\ensuremath{\mathbb{N}^{+}}} % set of Natural +
\newcommand{\hSoZ}  {\ensuremath{\mathbb{Z}}}      % set of Integers
\newcommand{\hSoZp} {\ensuremath{\mathbb{Z}^{+}}} % set of Integers +
\newcommand{\hSoZn} {\ensuremath{\mathbb{Z}^{-}}} % set of Integers -
\newcommand{\hSoZnn}{\ensuremath{\mathbb{Z}_{\ge 0}}} % set of Integers non negative
\newcommand{\hSoZnz}{\ensuremath{\mathbb{Z}_{\neq 0}}} % set of Real +
\newcommand{\hSoZnp}{\ensuremath{\mathbb{Z}_{\le 0}}} % set of Integers non positive
\newcommand{\hSoQ}  {\ensuremath{\mathbb{Q}}}      % set of Rationals 
\newcommand{\hSoQp} {\ensuremath{\mathbb{Q}^{+}}} % set of Rationals +
\newcommand{\hSoQn} {\ensuremath{\mathbb{Q}^{-}}} % set of Rationals -
\newcommand{\hSoQnn}{\ensuremath{\mathbb{Q}_{\ge 0}}} % set of Rationals non negative
\newcommand{\hSoQnz}{\ensuremath{\mathbb{Q}_{\neq 0}}} % set of Real +
\newcommand{\hSoQnp}{\ensuremath{\mathbb{Q}_{\le 0}}} % set of Rationals non positive
\newcommand{\hSoR}  {\ensuremath{\mathbb{R}}}      % set of Real Numbers
\newcommand{\hSoRp} {\ensuremath{\mathbb{R}^{+}}} % set of Real +
\newcommand{\hSoRn} {\ensuremath{\mathbb{R}^{-}}} % set of Real -
\newcommand{\hSoRnn}{\ensuremath{\mathbb{R}_{\ge 0}}} % set of Reals non negative
%\newcommand{\hSoRnp}{\ensuremath{\mathbb{R}_{\ge 0}}} % set of Reals non positive
\newcommand{\hSoRnz}{\ensuremath{\mathbb{R}_{\neq 0}}} % set of Real +
\newcommand{\hSoRnp}{\ensuremath{\mathbb{R}_{\le 0}}} % set of Reals non positive
\newcommand{\hSoC}  {\ensuremath{\mathbb{C}}}      % set of Complex Numbers
\newcommand{\hSoCnz}{\ensuremath{\mathbb{C}_{\neq 0}}} % set of nonzero Complex Numbers
%
\newcommand{\hSoPrimes}{\ensuremath{\mathbb{P}}} % set of Prime Numbers
\newcommand{\hSoTruth}{\ensuremath{\left \{ T, F \right \}}} % {T, F}
%\newcommand{\hSoBinary}{\ensuremath{\left \{ 0, 1 \right \}}} % {0, 1}
\newcommand{\hSoBits}{\ensuremath{\mathbb{B}}} % set of Bits
%
\newcommand{\hSoFunctions}[2]{\ensuremath{#2^{#1}}} % set of functions from 1to2
\newcommand{\hSoSubsets}[1]{\ensuremath{2^{#1}}} % set of subsets of 1
\newcommand{\hSoPowerSet}[1]{\ensuremath{\mathcal{P}(#1)}} % power set of 1
% ~~~~~~~~~~~~~~~~~~~~~~~~~~~~~~~~~~~~~~~ A




% ~~~~~~~~~~~~~~~~~~~~~~~~~~~~~~~~~~~~~~~ 
\begin{document}

\raggedright
\footnotesize




% ~~~~~~~~~~~~~~~~~~~~~~~~~~~~~~~~~~~~~~~ 
\begin{multicols}{3} % ~~~~~~~~~~~~~~~~~~~~~~~~~~~~~~~~~~~~~~~  column-3
	\setlength{\columnseprule}{0.25pt}
	\setlength{\premulticols}{1pt}
	\setlength{\postmulticols}{1pt}
	\setlength{\multicolsep}{1pt}
	\setlength{\columnsep}{2pt}




% ~~~~~~~~~~~~~~~~~~~~~~~~~~~~~~~~~~~~~~~ V title
\begin{center}
	\Large{\textbf{A Cheat Sheet for Discrete Math}}\\
	Haluk O. Bingol\\
	\small{
		\href
		     	{bingol@boun.edu.tr}
		     	{bingol@boun.edu.tr}
	}\\
	\small{
		\href
			{https://github.com/halukbingol/zintCheatSheet-LaTeX-DiscreteMath}
			{https://github.com/halukbingol/zintCheatSheet-LaTeX-DiscreteMath}
	}\\
	\small{\hbTimeStamp}
\end{center}
% ~~~~~~~~~~~~~~~~~~~~~~~~~~~~~~~~~~~~~~~ A




% ~~~~~~~~~~~~~~~~~~~~~~~~~~~~~~~~~~~~~~~ V Extending
\section{Extending \LaTeX}
\label{sec:extending}

\begin{verbatim}
	\newcommand{\mycmd}[1]{#1}}
	\operatorname*{myop}(x)
\end{verbatim}
% ~~~~~~~~~~~~~~~~~~~~~~~~~~~~~~~~~~~~~~~ A Extending




% ~~~~~~~~~~~~~~~~~~~~~~~~~~~~~~~~~~~~~~~ V Basics
\section{Basics}
\label{sec:basics}

\begin{tabular}{@{}ll@{}}
	$a, \mathbf{a}$	&\verb!a, \mathbf{a}!\\
	$\Phi, \boldsymbol{\Phi}$	&\verb!\Phi, \boldsymbol{\Phi}!\\
	$a^{ij}$	&\verb!a^{ij}!\\
	$a_{ij}$	&\verb!a_{ij}!\\
	$a_{i}^{j}$	&\verb!a_{i}^{j}!\\
	${}_{k}^{l}a_{i}^{j}$	&\verb!{}_{k}^{l}a_{i}^{j}!\\
%	$\head x$	&\verb!\dot x!\\
	$\bar x$	&\verb!\bar x!\\
	$\vec x$	&\verb!\vec x!\\
	$\dot x$	&\verb!\dot x!\\
	$\tilde{a}$	&\verb!\tilde{a}!\\
	%
	$\lvert a \rvert$ (\refsec{sec:pairs})	&\verb!\lvert a \rvert!\\
	$\lceil a \rceil$ (\refsec{sec:pairs})	&\verb!\lceil a \rceil!\\
	$\lfloor a \rfloor$ (\refsec{sec:pairs})	&\verb!\lfloor a \rfloor!\\
	%
	[2pt]
	$\overline{ab}$	&\verb!\overline{ab}!\\
	$\underline{ab}$	&\verb!\underline{ab}!\\
	[2pt]
	$\overrightarrow{AB}$	&\verb!\overrightarrow{AB}!\\
	[2pt]
	$\overleftarrow{AB}$	&\verb!\overleftarrow{AB}!\\
	[2pt]
	$\widetilde{ab}$	&\verb!\widetilde{ab}!\\
	[2pt]
	$\overbrace{a_{1} \cdots a_{k}}^{k}$	&\verb!\overbrace{a_{1} \cdots a_{k}}^{k}!\\
	[2pt]
	$\underbrace{a_{1} \cdots a_{k}}_{k}$	&\verb!\underbrace{a_{1} \cdots a_{k}}_{k}!\\
	[2pt]
	$M^{\top}$	~~= transpose	&\verb!M^{\top}!\\
	$\ell_{\bot}$	&\verb!\ell_{\bot}!\\
	$\ell_{\parallel}$	&\verb!\ell_{\parallel}!\\
%	$\underbracket{x}_{\text{real}}$	&\verb!aaa!\\
%	$\underbrace{x}_\text{real}$	&\verb!aaa!\\
%	$aaa$	&\verb!aaa!\\
%	$aaa$	&\verb!aaa!\\
%	$aaa$	&\verb!aaa!\\
%	$aaa$	&\verb!aaa!\\
\end{tabular}
% ~~~~~~~~~~~~~~~~~~~~~~~~~~~~~~~~~~~~~~~ A Basics




% ~~~~~~~~~~~~~~~~~~~~~~~~~~~~~~~~~~~~~~~ V Cancellation
\section{Cancellation}
\label{sec:cancellation}

\begin{verbatim}
	\usepackage{cancel}
\end{verbatim}

\begin{tabular}{@{}ll@{}}
	$a \neq b$	&\verb!a \neq b!\\
	$a \not\subseteq b$	&\verb!a \not\subseteq b!\\
	$\not aa$	&\verb!\not aa!\\
	$\not{aa}$	&\verb!\not{aa}!\\
	$\cancel{a b c \beta}$	&\verb!\cancel{a b c \beta}!\\	% req. cancel
	$\bcancel{a b c \beta}$	&\verb!\bcancel{a b c \beta}!\\	% req. cancel
	$\xcancel{a b c \beta}$	&\verb!\xcancel{a b c \beta}!\\	% req. cancel
	$\cancelto{\infty}{a b c \beta}$	&\verb!\cancelto{\infty}{a b c \beta}!\\	% req. cancel
%	$aaa$	&\verb!aaa!\\
\end{tabular}
% ~~~~~~~~~~~~~~~~~~~~~~~~~~~~~~~~~~~~~~~ V Cancellation




% ~~~~~~~~~~~~~~~~~~~~~~~~~~~~~~~~~~~~~~~ V Meta
\section{Meta}
\label{sec:meta}

\begin{tabular}{@{}ll@{}}
	$x \implies y$	&\verb!x \implies y!\\
	$x \impliedby y$	&\verb!x \impliedby y!\\
	$x \iff y$	&\verb!x \iff y!\\
	$x \Longrightarrow y$	&\verb!x \Longrightarrow y!\\
	$x \Longleftarrow y$	&\verb!x \Longleftarrow y!\\
	$a \triangleq b$	&\verb!a \triangleq b!\\
	$a \overset{\triangle}{\equiv} b$	&\verb!a \overset{\triangle}{\equiv} b!\\
	$a \overset{\triangle}{\iff} b$	&\verb!a \overset{\triangle}{\iff} b!\\
	$a \overset{\triangle}{\longleftrightarrow} b$
		&\verb!a \overset{\triangle}{\longleftrightarrow} b!\\
%	$aaa$	&\verb!aaa!\\
%	$aaa$	&\verb!aaa!\\
%	$aaa$	&\verb!aaa!\\
\end{tabular}
% ~~~~~~~~~~~~~~~~~~~~~~~~~~~~~~~~~~~~~~~ A Meta




% ~~~~~~~~~~~~~~~~~~~~~~~~~~~~~~~~~~~~~~~ V More
\section{More}

\begin{tabular}{@{}ll@{}}
	%
	$\sideset{_{a}^{b}}{_{c}^{d}} \sum$	
		&\verb!\sideset{_{a}^{b}}{_{c}^{d}} \sum!\\[2pt]
	%
	$\overset{ab}{xyz}$	
		&\verb!\overset{ab}{xyz}!\\[2pt]
	%
	$\underset{i}{\arg \, \min} \, \{ a_{i} \}$	
		&\verb!\underset{i}{\arg \, \min} \, \{ a_{i} \}!\\
	%
	$\underset{a}{\arg \, \max} \, f(a)$	
		&\verb!\underset{a}{\arg \, \max} \, f(a)!\\[2pt]
	%
	$\frac{a}{b}$	&\verb!\frac{a}{b}!\\[2pt]
	%
	$\tfrac{a}{b}$	&\verb!\tfrac{a}{b}!\\[2pt]
	%
	$\sum_{i = 0}^{5} i$	&\verb!\sum_{i = 0}^{5} i!\\[2pt]
	%
	$\prod_{i = 0}^{5} i$	&\verb!\prod_{i = 0}^{5} i!\\[2pt]
	%
	$\lim_{a \to \infty} x$	
		&\verb!\lim_{a \to \infty} x!\\[2pt]
	%
	$\frac{\mathrm{d} f(x)}{\mathrm{d} x}$	
		&\verb!\frac{\mathrm{d} f(x)}{\mathrm{d} x}!\\[2pt]
	%
	$a + \mathrm{i}\, b$
		&\verb~a + \mathrm{i}\, b~\\[2pt]
	%
	$\int_{a}^{b} \! f(x) \, \mathrm{d} x$	
		&\verb~\int_{a}^{b} \! f(x) \, \mathrm{d} x~\\[2pt]
	%
	$\left[\frac{\infty}{\infty}\right]$ 
		&\verb~\left[\frac{\infty}{\infty}\right]~\\
%\makeatletter
%\renewcommand\d[1]{\mspace{6mu}\mathrm{d}#1\@ifnextchar\d{\mspace{-3mu}}{}}
%\makeatother
%	$aaa$	&\verb!aaa!\\
%	$aaa$	&\verb!aaa!\\
%	$aaa$	&\verb!aaa!\\
%	$aaa$	&\verb!aaa!\\
\end{tabular}
% ~~~~~~~~~~~~~~~~~~~~~~~~~~~~~~~~~~~~~~~ V More




% ~~~~~~~~~~~~~~~~~~~~~~~~~~~~~~~~~~~~~~~ V Format Patterns}
\section{Format Patterns}

% ~~~~~~~~~~~~~~~~~~~~~~~~~~~~~~~~~~~~~~~ -
%\begin{tabular}{@{}ll@{}}
%	a
%	\begin{tabular}{|l|}
%		b \\
%		c
%	\end{tabular}
%	d.\\
%&\verb!a
%	\begin{tabular}{|l|}
%		b \\
%		c
%	\end{tabular}
%	d.\\
%!\\
%\end{tabular}\\

% ~~~~~~~~~~~~~~~~~~~~~~~~~~~~~~~~~~~~~~~ -
\hSep
\begin{multicols}{2}
		a
		\begin{tabular}{|l|}
		  bb \\
		  cc
		\end{tabular}
		d.
\columnbreak
	\begin{verbatim}
		a
		\begin{tabular}{|l|}
		  bb \\
		  cc
		\end{tabular}
		d.
	\end{verbatim}
\end{multicols}
% ~~~~~~~~~~~~~~~~~~~~~~~~~~~~~~~~~~~~~~~ -
\hSep
\begin{multicols}{2}
	 $
		{\genfrac(]{0pt}{2}
		  {a+b}
		  {c+d+e}
		}
	 $
\columnbreak
	\begin{verbatim}
		{\genfrac(]{0pt}{2}
		  {a+b}
		  {c+d+e}
		}
	\end{verbatim}
\end{multicols}
% ~~~~~~~~~~~~~~~~~~~~~~~~~~~~~~~~~~~~~~~ -
\hSep
\begin{multicols}{2}
	$
		a = 
		\begin{cases}
		  1, & n \text{ is odd}, \\
		  0, & \text{otherwise}.
		\end{cases}
	$
\columnbreak
	\begin{verbatim}
		a = 
		\begin{cases}
		  1, & n \text{ is odd}, \\
		  0, & \text{otherwise}.
		\end{cases}
	\end{verbatim}
\end{multicols}
% ~~~~~~~~~~~~~~~~~~~~~~~~~~~~~~~~~~~~~~~ -
\hSep
\begin{multicols}{2}
	\[
		\sum_{
		  \substack{
		    k \in \hSoZ\\ 
		    7 < k \\ 
		    k \leq 4
		  }
		} 
		a_{k}.
	\]
\columnbreak
	\begin{verbatim}
		\sum_{
		  \substack{
		    k \in \hSoZ\\ 
		    7 < k \\ 
		    k \leq 4
		  }
		} 
		a_{k}.
	\end{verbatim}
\end{multicols}
% ~~~~~~~~~~~~~~~~~~~~~~~~~~~~~~~~~~~~~~~ -
\hSep
\begin{multicols}{2}
	\[
		n! = 
		\underbrace{
		  1
		  \cdot 2 
		  \cdot 
		  \dotso 
		  \cdot n
		}_{n}
	\]
\columnbreak
	\begin{verbatim}
		n! = 
		\underbrace{
		  1
		  \cdot 2 
		  \cdot 
		  \dotso 
		  \cdot n
		}_{n}
	\end{verbatim}
\end{multicols}
% ~~~~~~~~~~~~~~~~~~~~~~~~~~~~~~~~~~~~~~~ -
\hSep
\begin{multicols}{2}
		\begin{align*}
			aaa &= b+b+b &// ccc\\
			d &= e+e &// fff
		\end{align*}
\columnbreak
	\begin{verbatim}
		\begin{align*}
			aaa &= b+b+b &// ccc\\
			d &= e+e &// fff
		\end{align*}
	\end{verbatim}
\end{multicols}
\hSep
% ~~~~~~~~~~~~~~~~~~~~~~~~~~~~~~~~~~~~~~~ A Format Patterns}




% ~~~~~~~~~~~~~~~~~~~~~~~~~~~~~~~~~~~~~~~ V Equations
\section{Equations}

% ~~~~~~~~~~~~~~~~~~~~~~~~~~~~~~~~~~~~~~~ -
\hSep
\begin{multicols}{2}
	\begin{align*}
	  1 + (2 + 3)
	  & = 1 + 5\\
	  & = 6\\
	  & = 12/2.
	\end{align*}
\columnbreak
	\begin{verbatim}
	\begin{align*}
	  1 + (2 + 3)
	  & = 1 + 5\\
	  & = 6\\
	  & = 12/2.
	\end{align*}
	\end{verbatim}
\end{multicols}
% ~~~~~~~~~~~~~~~~~~~~~~~~~~~~~~~~~~~~~~~ -
\hSep
\begin{multicols}{2}
	\[
		\left.
		  \frac{x^{2}}{3}
		\right|_{0}^{1}
	\]
\columnbreak
	\begin{verbatim}
		\left.
		  \frac{x^{2}}{3}
		\right|_{0}^{1}
	\end{verbatim}
\end{multicols}
%% ~~~~~~~~~~~~~~~~~~~~~~~~~~~~~~~~~~~~~~~ -
%\hSep
%\begin{multicols}{2}
%	aaa
%\columnbreak
%	\begin{verbatim}
%		aaa
%	\end{verbatim}
%\end{multicols}
\hSep
% ~~~~~~~~~~~~~~~~~~~~~~~~~~~~~~~~~~~~~~~ A Equations




% ~~~~~~~~~~~~~~~~~~~~~~~~~~~~~~~~~~~~~~~ V Spaces
\section{Spaces in Math mode}

\begin{tabular}{@{}ll@{}}
	$a\!a$	&\verb!a\!a! negative\\
	$a\,a$	&\verb!a\,a! thin\\
	$a\:a$	&\verb!a\:a! medium\\
	$a\;a$	&\verb!a\;a! thick\\
	$a\ a$	&\verb!a\ a! ?\\
	$a\quad a$	&\verb!a\quad a!\\
	$a\qquad a$	&\verb!a\qquad a!\\
%	$aaa$	&\verb!aaa!\\
%	$aaa$	&\verb!aaa!\\
%	$aaa$	&\verb!aaa!\\
\end{tabular}
% ~~~~~~~~~~~~~~~~~~~~~~~~~~~~~~~~~~~~~~~ A Spaces




% ~~~~~~~~~~~~~~~~~~~~~~~~~~~~~~~~~~~~~~~ V Dots
\section{Dots}

\begin{tabular}{@{}ll@{}}
	$1 \cdot 2 \cdot 3$ (multiplication) & \verb!$1 \cdot 2 \cdot 3$!\\
	$1, 2, \dotsc, 9$ (comma) & \verb!$1, 2, \dotsc, 9$!\\
	$1 + 2 + \dotso + 9$ (operator) & \verb!$1 + 2 + \dotso + 9$!\\
%	$aaa$	&\verb!aaa!\\
%	$aaa$	&\verb!aaa!\\
\end{tabular}
% ~~~~~~~~~~~~~~~~~~~~~~~~~~~~~~~~~~~~~~~ A Dots




% ~~~~~~~~~~~~~~~~~~~~~~~~~~~~~~~~~~~~~~~ V Matrices
\section{Matrices}

% ~~~~~~~~~~~~~~~~~~~~~~~~~~~~~~~~~~~~~~~ 
\hSep
\begin{multicols}{2}
	\[
		\begin{pmatrix}
		  a_{1,1} & \cdots & a_{1,n}\\
		  \vdots  & \ddots & \vdots\\
		  a_{m,1} & \cdots & a_{m,n}
		\end{pmatrix}
	\]
\columnbreak
	\begin{verbatim}
		\begin{pmatrix}
		  a_{1,1} & \cdots & a_{1,n}\\
		  \vdots  & \ddots & \vdots\\
		  a_{m,1} & \cdots & a_{m,n}
		\end{pmatrix}
	\end{verbatim}
\end{multicols}
% ~~~~~~~~~~~~~~~~~~~~~~~~~~~~~~~~~~~~~~~ 
\hSep
\begin{multicols}{2}
	\[
		\bordermatrix{
		  ~ & x & y \cr
		  A & 1 & 2 \cr
		  B & 3 & 4 \cr
		}
	\]
\columnbreak
	\begin{verbatim}
		\bordermatrix{
		  ~ & x & y \cr
		  A & 1 & 2 \cr
		  B & 3 & 4 \cr
		}
	\end{verbatim}
\end{multicols}
%% ~~~~~~~~~~~~~~~~~~~~~~~~~~~~~~~~~~~~~~~ 
%\hSep
%\begin{multicols}{2}
%	aaa
%\columnbreak
%	\begin{verbatim}
%	aaa
%	\end{verbatim}
%\end{multicols}
\hSep
% ~~~~~~~~~~~~~~~~~~~~~~~~~~~~~~~~~~~~~~~ V Matrices




% ~~~~~~~~~~~~~~~~~~~~~~~~~~~~~~~~~~~~~~~ V Logic
\section{Logic}

\begin{tabular}{@{}ll@{}}
	$p \implies  q$	&\verb!p \implies  q!\\
	$p \impliedby  q$	&\verb!p \impliedby  q!\\
	$p \iff  q$	&\verb!p \iff  q!\\
	%
	$\overline{p}$	&\verb!\overline{p}!\\
	$\neg p$	&\verb!\neg p!\\
	$p \land q$	&\verb!p \land q!\\
	$p \lor q$	&\verb!p \lor q!\\
	$p \oplus q$	&\verb!p \oplus q!\\
	$p \rightarrow q$	&\verb!p \rightarrow q!\\
	$p \leftrightarrow q$	&\verb!p \leftrightarrow q!\\
	$p \equiv q$	&\verb!p \equiv q!\\
	%
	$p \longrightarrow q$	&\verb!p \longrightarrow q!\\
	$p \longleftrightarrow q$	&\verb!p \longleftrightarrow q!\\
%	$p \Longrightarrow  q$	&\verb!p \Longrightarrow  q!\\
%	$p \Longleftrightarrow  q$	&\verb!p \Longleftrightarrow  q!\\
	$\forall x \in A \ P(x)$	&\verb!\forall x \in A \ P(x)!\\
	$\not \forall x \in A \ P(x)$	&\verb!\not \forall x \in A \ P(x)!\\
	$\exists x \in A \ P(x)$	&\verb!\exists x \in A \ P(x)!\\
	$\not \exists x \in A \ P(x)$	&\verb!\not \exists x \in A \ P(x)!\\
%	$aaa$	&\verb!aaa!\\
%	$aaa$	&\verb!aaa!\\
%	$aaa$	&\verb!aaa!\\
\end{tabular}
% ~~~~~~~~~~~~~~~~~~~~~~~~~~~~~~~~~~~~~~~ V Logic




% ~~~~~~~~~~~~~~~~~~~~~~~~~~~~~~~~~~~~~~~ V Set Theory
\section{Set Theory}

\begin{tabular}{@{}ll@{}}
%	$\hPairing{b}{aa}$	&\verb!\hPairing{b}{aa}!\\
%	$\ensuremath{\left 1 \, 2 \,  2}$	&aaa\\
	$\emptyset$	&\verb!\emptyset!\\
	$x \in A$	&\verb!x \in A!\\
	$x \notin A$	&\verb!x \notin A!\\
	$\{ x, y \}$	& \verb!\{ x, y \}!\\
	$\{ x \mid P(x) \}$	&\verb!\{ x \mid P(x) \}!\\
	$A \subset B$	&\verb!A \subset B!\\
	$A \not\subset B$	&\verb!A \not\subset B!\\
	$A \subseteq B$	&\verb!A \subseteq B!\\
	$A \not\subseteq B$	&\verb!A \not\subseteq B!\\
	$2^{A}$ 	~~= power set of $A$	&\verb!2^{A}!\\
	$\mathcal{P}(A)$ 	~~= power set of $A$	&\verb!\mathcal{P}(A)!\\
	$\lvert A \rvert$  	~~= cardinality of $A$ (\refsec{sec:pairs})	&\verb!\lvert A \rvert!\\
	$A \cup B$	&\verb!A \cup B!\\
	$A \cap B$	&\verb!A \cap B!\\
	$A \smallsetminus B$	~~= set difference	&\verb!A \smallsetminus B!\\
	$A \times B$  	~~= Cartesian product	&\verb!A \times B!\\
	$(a, b)$ 	~~= ordered pair	&\verb!(a, b)!\\
%	
	$\overline{A}$	~~= complement of $A$	&\verb!\overline{A}!\\
	$f^{-1}$	~~= inverse	&\verb!f^{-1}!\\
	$f \circ g$	~~= composition	&\verb!f \circ g!\\
	$a \, \beta \, b$ 	~~= relation	&\verb!a \, \beta \, b!\\
	$a \, \cancel{\beta} \, b$	&\verb!a \,\cancel{\beta} \,b!\\
	$M_{\beta}$ 	~~= matrix of $\beta$	&\verb!aaa!\\
	$f \colon A \to B$	~~= function	&\verb!f \colon A \to B!\\
	$a \mapsto f(a)$~~= mapped to	&\verb!a \mapsto f(a)!\\
	$f \upharpoonright C$	~~= restriction of $f$ to $C$	
		&\verb!f \upharpoonright C!\\
	$i_{A}$	~~= identity function of $A$	&\verb!i_{A}!\\
%	$aaa$	&\verb!aaa!\\
	$B^{A}$	~~= set of all functions	&\verb!B^{A}!\\
	$\mathbb{A}$	&\verb!\mathbb{A}!\\
	$\mathcal{A}$	&\verb!\mathcal{A}!\\
%	$aaa$	&\verb!aaa!\\
%	$aaa$	&\verb!aaa!\\
%	$aaa$	&\verb!aaa!\\
\end{tabular}
% ~~~~~~~~~~~~~~~~~~~~~~~~~~~~~~~~~~~~~~~ A Set Theory




% ~~~~~~~~~~~~~~~~~~~~~~~~~~~~~~~~~~~~~~~ V Algebraic Structures
\section{Algebraic Structures}

\begin{tabular}{@{}ll@{}}
	$[A, \oplus]$	&\verb![A, \oplus]!\\
	$[A, \oplus, \otimes]$	&\verb![A, \oplus, \otimes]!\\
%	$aaa$	&\verb!aaa!\\
\end{tabular}
% ~~~~~~~~~~~~~~~~~~~~~~~~~~~~~~~~~~~~~~~ A Algebraic Structures




% ~~~~~~~~~~~~~~~~~~~~~~~~~~~~~~~~~~~~~~~ V Number Theory
\section{Number Theory}

\begin{tabular}{@{}ll@{}}
	$\hCeil{x}$ &\verb!\hCeil{x}!\\
	$\hFloor{x}$ &\verb!\hPairingFloor{x}!\\
	$\hAbs{x}$ &\verb!\hAbs{x}!\\
	$x \mid y$ &\verb!x \mid y!\\
	$x \nmid y$ &\verb!x \nmid y!\\
	$x \bot y$ &\verb!x \bot y!\\
	$x \perp y$	&\verb!x \perp y!\\
	$x \ \mathrm{div} \ y$ &\verb!x \ \mathrm{div} \ y!\\
	$x \ \mathrm{rem} \ y$ &\verb!x \ \mathrm{rem} \ y!\\
	$\log_{2} x$ &\verb!\log_{2} x!\\[2pt]
%	$H_{2} x$ &\verb!H_{2} x!\\
%	$\hBinaryEntropy$ &\verb!\hBinaryEntropy!\\
	$\sum_{i=1}^{n} a_{i}$ &\verb!\sum_{i=1}^{n} a_{I}!\\[2pt]
	$\prod_{i=1}^{n} a_{i}$ &\verb!\prod_{i=1}^{n} a_{I}!\\[2pt]
	$\frac{x}{y}$ &\verb!\frac{x}{y}!\\[2pt]
	$\sqrt[n]{x}$ &\verb!\sqrt[n]{x}!\\
	$a \bmod b$ &\verb!a \bmod b!\\
	$0 \equiv 3 \pmod{3}$ &\verb!0 \equiv 3 \pmod{3}!\\
	$0 \equiv 3 \mod{3}$  &\verb!0 \equiv 3 \mod{3}!\\
	$0 \equiv 3 \pod{3}$ &\verb!0 \equiv 3 \pod{3}!\\
%	$aaa$ &\verb!aaa!\\
%	$aaa$ &\verb!aaa!\\
\end{tabular}
% ~~~~~~~~~~~~~~~~~~~~~~~~~~~~~~~~~~~~~~~ A Number Theory




% ~~~~~~~~~~~~~~~~~~~~~~~~~~~~~~~~~~~~~~~ V Combinatorics
\section{Combinatorics}

\begin{tabular}{@{}ll@{}}
	$n! = n$ factorial	&\verb!n!!\\[2pt]
	$n^{\overline{r}}$ = $n$ to the $r$ rising	&\verb!n^{\overline{r}}!\\[2pt]
	$n^{\underline{r}}$ = $n$ to the $r$ falling	&\verb!n^{\underline{r}}!\\[2pt]
	$\binom{n}{r}$ = $n$ choose $r$	&\verb!\binom{n}{r}!\\[2pt]
	${n \choose r}$ = $n$ choose $r$	&\verb!{n \choose r}!\\[2pt]
	${n \brack r}$ = Stirling cycle number	&\verb!{n \brack r}!\\[2pt]
	${n \brace r}$ = Stirling subset number &\verb!{n \brace r}!\\[2pt]
	$S(n, k)$ = Stirling number	&\verb!S(n, k)!\\
	%=The number of partitions of an $n$-set into exactly $k$ nonempty subsets. \\
	$B_{n}$ = The $n$th Bell number	&\verb!B_{n}!\\
	%=The number of partitions of an $n$-set. \\
\end{tabular}
% ~~~~~~~~~~~~~~~~~~~~~~~~~~~~~~~~~~~~~~~ A Combinatorics




% ~~~~~~~~~~~~~~~~~~~~~~~~~~~~~~~~~~~~~~~ 
\section{ $h$ extensions}

\begin{verbatim}
	See the source of this document.
\end{verbatim}




% ~~~~~~~~~~~~~~~~~~~~~~~~~~~~~~~~~~~~~~~ V hTags
\subsection{$h$Tags}
\label{sec:hTags}

%\begin{verbatim}
%	\newcommand{\hDed}[1]{\textcolor{darkred}{\textit{#1}}}
%	\newcommand{\hDiff}{\ensuremath{\overset{\Delta}{\iff}}}
%	\newcommand{\hDeq}{\ensuremath{\ \triangleq} \ }
%	\newcommand{\hDev}{\ensuremath{\ \overset{\Delta}{\equiv}}\ }
%\end{verbatim}

\begin{tabular}{@{}ll@{}}
	aa \hDed{dd} bb.	&\verb!aa \hDed{dd} bb.!\\
	$x \hDiff  y$	&\verb!x \hDiff y!\\
	$x \hDeq y$	&\verb!x \hDeq y!\\
	$x \hDev y$	&\verb!x \hDev y!\\
%	$aaa$	&\verb!aaa!\\
%	$aaa$	&\verb!aaa!\\
%	$aaa$	&\verb!aaa!\\
\end{tabular}
% ~~~~~~~~~~~~~~~~~~~~~~~~~~~~~~~~~~~~~~~ A hTags




% ~~~~~~~~~~~~~~~~~~~~~~~~~~~~~~~~~~~~~~~ V hPairs
\subsection{$h$Pairs}
\label{sec:pairs}

%\begin{verbatim}
%	\newcommand{\hAbs}[1]{\left\lvert\, #1\, \right\rvert} 
%	\newcommand{\hCeil}[1]{\left\lceil\, #1\, \right\rceil}
%	\newcommand{\hFloor}[1]	{\left\lfloor\, #1\, \right\rfloor}
%\end{verbatim}

\begin{tabular}{@{}ll@{}}
	$\hAbs{x}$	&\verb!\hAbs{x}!\\
	$\hCeil{x}$	&\verb!\hCeil{x}!\\
	$\hFloor{x}$	&\verb!\hFloor{x}!\\
	$\| x \|$	&\verb!\| x \|!\\
%	$aaa$	&\verb!aaa!\\
%	$aaa$	&\verb!aaa!\\
%	$aaa$	&\verb!aaa!\\
\end{tabular}
% ~~~~~~~~~~~~~~~~~~~~~~~~~~~~~~~~~~~~~~~ A hPairs




% ~~~~~~~~~~~~~~~~~~~~~~~~~~~~~~~~~~~~~~~ V Sets
\subsection{$h$Sets}

\begin{tabular}{@{}ll@{}}
	\hSoN = The natural numbers	&\verb!\hSoN!\\
	\hSoNp = Counting numbers	&\verb!\hSoNp!\\
	\hSoZ = The integers	&\verb!\hSoZ!\\
	\hSoZp = positive integers	&\verb!\hSoZp!\\
	\hSoZn = negative integers	&\verb!\hSoZn! \\
	\hSoZnn = nonnegative integers	&\verb!\hSoZnn!\\
	\hSoZnz = nonzero integers	&\verb!\hSoZnz!\\
	\hSoZnp = nonpositive integers	&\verb!\hSoZnp!\\
	\hSoQ = The rational numbers	&\verb!\hSoQ!\\
	\hSoQp = positive rational numbers	&\verb!\hSoQp!\\
	\hSoQn = negative rational numbers	&\verb!\hSoQn!\\
	\hSoQnn = nonnegative rationals	&\verb!\hSoQnn!\\
	\hSoQnz = nonzero rationals	&\verb!\hSoQnz!\\
	\hSoQnp = nonpositive rationals	&\verb!\hSoQnp!\\
	\hSoR = The real numbers	&\verb!\hSoR!\\
	\hSoRp = positive real numbers	&\verb!\hSoRp!\\
	\hSoRn = negative real numbers	&\verb!\hSoRn!\\
	\hSoRnn = nonnegative reals	&\verb!\hSoRnn!\\
	\hSoRnz = nonzero reals	&\verb!\hSoRnz!\\
	\hSoRnp = nonpositive reals	&\verb!\hSoRnp!\\
	\hSoC = The complex numbers	&\verb!\hSoC!\\
	\hSoCnz = nonzero complex numbers	&\verb!\hSoCnz!\\
	\hSoPrimes = The set of Prime Numbers	&\verb!\hSoPrimes!\\
	\hSoTruth	&\verb!\hSoTruth! \\
	$\hSoBits = \{ 0, 1 \}$	&\verb!\hSoBits!\\
	$\hSoSubsets{A}$	&\verb!\hSoSubsets{A}!\\
	$\hSoPowerSet{A}$ = $\hSoSubsets{A}$= The set of all subsets	&\verb!\hSoPowerSet{A}! \\
	$\hSoFunctions{A}{B}$=The set of all functions	&\verb!\hSoFunctions{A}{B}!\\
%	$aaa$	&\verb!aaa!\\
%	$aaa$	&\verb!aaa!\\
%	$aaa$	&\verb!aaa!\\
%	$aaa$	&\verb!aaa!\\
\end{tabular}
% ~~~~~~~~~~~~~~~~~~~~~~~~~~~~~~~~~~~~~~~ A Sets




% ~~~~~~~~~~~~~~~~~~~~~~~~~~~~~~~~~~~~~~~ V Labels References
\subsection{Labels References}

%\begin{verbatim}
%	\newcommand{\reffig}[1]{Fig.~\ref{#1}}
%	\newcommand{\reftbl}[1]{Table~\ref{#1}}
%	\newcommand{\refsec}[1]{Sec.~\ref{#1}}
%	\newcommand{\refthm}[1]{Theorem~\ref{#1}}
%	\newcommand{\reflem}[1]{Lemma~\ref{#1}}
%	\newcommand{\refdef}[1]{Definition~\ref{#1}}
%	\newcommand{\refeq}[1]{Eq.~\ref{#1}}
%\end{verbatim}

\begin{tabular}{@{}ll@{}}
	$\refdef{def:aa}$	&\verb!\refdef{def:aa}!\\
	$\refthm{thm:gauss}$	&\verb!\refthm{thm:gauss}!\\
	$\reflem{lem:fermat}$	&\verb!\reflem{lem:fermat}!\\
	$\refsec{sec:templates}$	&\verb!\refsec{sec:templates}!\\
%	$aaa$	&\verb!aaa!\\
%	$aaa$	&\verb!aaa!\\
%	$aaa$	&\verb!aaa!\\
\end{tabular}
% ~~~~~~~~~~~~~~~~~~~~~~~~~~~~~~~~~~~~~~~ A Labels References




% ~~~~~~~~~~~~~~~~~~~~~~~~~~~~~~~~~~~~~~~ V Math Environments
\section{Math Environments}

\begin{tabular}{@{}ll@{}}
	Axiom	&\verb!ax!\\
	Corollary	&\verb!cor!\\
	Definition	&\verb!defn!\\
	Example	&\verb!exmp!\\
	Exercise	&\verb!exercise!\\
	Equation	&\verb!align*!\\
	Lemma	&\verb!lem!\\
	Notation	&\verb!notation!\\
	Proof	&\verb!proof!\\
	Proposition	&\verb!prop!\\
	Remark	&\verb!rem!\\
	Theorem	&\verb!thm!\\
%	Question	&\verb!hQuestion!\\
%	Reminder	&\verb!hReminder!\\
%	Property	&\verb!hProperty!\\
%	Notation	&\verb!hNotation!\\
%	Application	&\verb!hApplication!\\
%	Summary	&\verb!hSummary!\\
\end{tabular}
% ~~~~~~~~~~~~~~~~~~~~~~~~~~~~~~~~~~~~~~~ A Math Environments




\columnbreak % ~~~~~~~~~~~~~~~~~~~~~~~~~~~~~~~~~~~~~~~  column-3




% ~~~~~~~~~~~~~~~~~~~~~~~~~~~~~~~~~~~~~~~  V Environment Usage
\section{Environment Usage}
\label{sec:templates}

% ~~~~~~~~~~~~~~~~~~~~~~~~~~~~~~~~~~~~~~~ -
\hSep
\begin{multicols}{2}
		\begin{ax}
		  aaa
		  \label{ax:one}
		\end{ax}
\columnbreak
	\begin{verbatim}
		\begin{ax}
		  aaa
		  \label{ax:one}
		\end{ax}
	\end{verbatim}
\end{multicols}
% ~~~~~~~~~~~~~~~~~~~~~~~~~~~~~~~~~~~~~~~ -
\hSep
\begin{multicols}{2}
		\begin{defn}
		  aaa \hDed{ddd} bbb.
		  \label{def:aa}
		\end{defn}
\columnbreak
	\begin{verbatim}
		\begin{defn}
		  aaa \hDed{ddd} bbb.
		  \label{def:aa}
		\end{defn}
	\end{verbatim}
\end{multicols}
% ~~~~~~~~~~~~~~~~~~~~~~~~~~~~~~~~~~~~~~~ -
\hSep
\begin{multicols}{2}
		\begin{thm}[Gauss]
		  aaa
		  \label{thm:gauss}
		\end{thm}
\columnbreak
	\begin{verbatim}
		\begin{thm}[Gauss]
		  aaa
		  \label{thm:gauss}
		\end{thm}
	\end{verbatim}
\end{multicols}
% ~~~~~~~~~~~~~~~~~~~~~~~~~~~~~~~~~~~~~~~ -
\hSep
\begin{multicols}{2}
		\begin{proof}
		  aaa
		\end{proof}
\columnbreak
	\begin{verbatim}
		\begin{proof}
		  aaa
		\end{proof}
	\end{verbatim}
\end{multicols}
% ~~~~~~~~~~~~~~~~~~~~~~~~~~~~~~~~~~~~~~~ -
\hSep
\begin{multicols}{2}
		\begin{lem}[Fermat]
		  aaa
		  \label{lem:fermat}
		\end{lem}
\columnbreak
	\begin{verbatim}
		\begin{lem}[Fermat]
		  aaa
		  \label{lem:fermat}
		\end{lem}
	\end{verbatim}
\end{multicols}
% ~~~~~~~~~~~~~~~~~~~~~~~~~~~~~~~~~~~~~~~ -
\hSep
\begin{multicols}{2}
		\begin{cor}
		  aaa
		\end{cor}
\columnbreak
	\begin{verbatim}
		\begin{cor}
		  aaa
		\end{cor}
	\end{verbatim}
\end{multicols}
% ~~~~~~~~~~~~~~~~~~~~~~~~~~~~~~~~~~~~~~~ -
\hSep
\begin{multicols}{2}
		\begin{prop}
		  aaa
		\end{prop}
\columnbreak
	\begin{verbatim}
		\begin{prop}
		  aaa
		\end{prop}
	\end{verbatim}
\end{multicols}
% ~~~~~~~~~~~~~~~~~~~~~~~~~~~~~~~~~~~~~~~ -
\hSep
\begin{multicols}{2}
		\begin{rem}
		  aaa
		\end{rem}
\columnbreak
	\begin{verbatim}
		\begin{rem}
		  aaa
		\end{rem}
	\end{verbatim}
\end{multicols}
% ~~~~~~~~~~~~~~~~~~~~~~~~~~~~~~~~~~~~~~~ -
\hSep
\begin{multicols}{2}
		\begin{notation}
		  aaa
		\end{notation}
\columnbreak
	\begin{verbatim}
		\begin{notation}
		  aaa
		\end{notation}
	\end{verbatim}
\end{multicols}
%% ~~~~~~~~~~~~~~~~~~~~~~~~~~~~~~~~~~~~~~~ -
%\hSep
%\begin{multicols}{2}
%		\begin{exmp}
%		  aaa
%		\end{exmp}
%\columnbreak
%	\begin{verbatim}
%		\begin{exmp}
%		  aaa
%		\end{exmp}
%	\end{verbatim}
%\end{multicols}
\hSep
% ~~~~~~~~~~~~~~~~~~~~~~~~~~~~~~~~~~~~~~~  A Environment Usage





% ~~~~~~~~~~~~~~~~~~~~~~~~~~~~~~~~~~~~~~~ 
\columnbreak % ~~~~~~~~~~~~~~~~~~~~~~~~~~~~~~~~~~~~~~~  column-3



% ~~~~~~~~~~~~~~~~~~~~~~~~~~~~~~~~~~~~~~~ V xymatrix
\section{xymatrix}

$
	\xymatrix{
	%
	11
	\POS[];
	[d]**\dir{-};
	[];[dr]**\dir{-};
	\ar@/^/[r]^{\alpha_{1}} 
	%
	& 12
	\ar@/^/[r]^{\alpha_{7}} 
	\ar@/^/[l]^{\alpha_{3}} 
	%
	& 13
	\ar@/^/[l]^{\alpha_{9}} 
	\\
	%
	21
	\ar@/^/[ur]^{\alpha_{5}}
	\ar@/^/[u]^{\alpha_{4}}
	\ar@/^/[urr]^{\alpha_{7}}
	%
	& \bullet
	%
	& 23
	\ar@/^/[ll]^{\alpha_{8}}
	\ar[u]^{9}
	\ar[ul]^{9}
	}
$
\begin{verbatim}
	\xymatrix{
	%
	11
	\POS[];
	[d]**\dir{-};
	[];[dr]**\dir{-};
	\ar@/^/[r]^{\alpha_{1}} 
	%
	& 12
	\ar@/^/[r]^{\alpha_{7}} 
	\ar@/^/[l]^{\alpha_{3}} 
	%
	& 13
	\ar@/^/[l]^{\alpha_{9}} 
	\\
	%
	21
	\ar@/^/[ur]^{\alpha_{5}}
	\ar@/^/[u]^{\alpha_{4}}
	\ar@/^/[urr]^{\alpha_{7}}
	%
	& \bullet
	%
	& 23
	\ar@/^/[ll]^{\alpha_{8}}
	\ar[u]^{9}
	\ar[ul]^{9}
	}
\end{verbatim}
% ~~~~~~~~~~~~~~~~~~~~~~~~~~~~~~~~~~~~~~~ V xymatrix




% ~~~~~~~~~~~~~~~~~~~~~~~~~~~~~~~~~~~~~~~ V Lists
\section{Lists}

% ~~~~~~~~~~~~~~~~~~~~~~~~~~~~~~~~~~~~~~~ 
\verb!\usepackage{enumerate}!
% ~~~~~~~~~~~~~~~~~~~~~~~~~~~~~~~~~~~~~~~ 
\hSep
\begin{multicols}{2}
		\begin{itemize}
		  \item aa
		  \item bb
		\end{itemize}
\columnbreak
	\begin{verbatim}
		\begin{itemize}
		  \item aa
		  \item bb
		\end{itemize}
	\end{verbatim}
\end{multicols}
% ~~~~~~~~~~~~~~~~~~~~~~~~~~~~~~~~~~~~~~~ -
\hSep
\begin{multicols}{2}
		\begin{enumerate}
		  \item aa
		  \item bb
		\end{enumerate}
\columnbreak
	\begin{verbatim}
		\begin{enumerate}
		  \item aa
		  \item bb
		\end{enumerate}
	\end{verbatim}
\end{multicols}
% ~~~~~~~~~~~~~~~~~~~~~~~~~~~~~~~~~~~~~~~ -
\hSep
\begin{multicols}{2}
		\begin{enumerate}[i)]
		  \item aa
		  \item bb
		\end{enumerate}
\columnbreak
	\begin{verbatim}
		\begin{enumerate}[i)]
		  \item aa
		  \item bb
		\end{enumerate}
	\end{verbatim}
\end{multicols}
% ~~~~~~~~~~~~~~~~~~~~~~~~~~~~~~~~~~~~~~~ -
\hSep
\begin{multicols}{2}
		\begin{enumerate}[a)]
		  \item aa
		  \item bb
		\end{enumerate}
\columnbreak
	\begin{verbatim}
		\begin{enumerate}[a)]
		  \item aa
		  \item bb
		\end{enumerate}
	\end{verbatim}
\end{multicols}
\hSep
% ~~~~~~~~~~~~~~~~~~~~~~~~~~~~~~~~~~~~~~~ A Lists




% ~~~~~~~~~~~~~~~~~~~~~~~~~~~~~~~~~~~~~~~ V Math mode
\section{Math mode}

% ~~~~~~~~~~~~~~~~~~~~~~~~~~~~~~~~~~~~~~~ 
\hSep
\begin{multicols}{2}
		inline: $aaa$  or \(aaa\) bbb\\
\columnbreak
	\begin{verbatim}
		inline: 
		$
		  aaa
		$  
		or 
		\(
		  aaa
		\) 
		bbb
	\end{verbatim}
\end{multicols}
% ~~~~~~~~~~~~~~~~~~~~~~~~~~~~~~~~~~~~~~~ 
\hSep
\begin{multicols}{2}
		display: \[ aaa\] bbb
\columnbreak
	\begin{verbatim}
		display: 
		\[
		  aaa
		\] 
		bbb
	\end{verbatim}
\end{multicols}
% ~~~~~~~~~~~~~~~~~~~~~~~~~~~~~~~~~~~~~~~ A




% ~~~~~~~~~~~~~~~~~~~~~~~~~~~~~~~~~~~~~~~ 
\end{multicols} % ~~~~~~~~~~~~~~~~~~~~~~~~~~~~~~~~~~~~~~~  column-3
% ~~~~~~~~~~~~~~~~~~~~~~~~~~~~~~~~~~~~~~~ 
\end{document}
